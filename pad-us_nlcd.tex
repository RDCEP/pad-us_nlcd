% Created 2013-03-27 Wed 18:02
\documentclass[11pt]{article}
\usepackage[utf8]{inputenc}
\usepackage[T1]{fontenc}
\usepackage{fixltx2e}
\usepackage{graphicx}
\usepackage{longtable}
\usepackage{float}
\usepackage{wrapfig}
\usepackage{soul}
\usepackage{textcomp}
\usepackage{marvosym}
\usepackage{wasysym}
\usepackage{latexsym}
\usepackage{amssymb}
\usepackage{hyperref}
\tolerance=1000
\providecommand{\alert}[1]{\textbf{#1}}

\title{Synthesis of a disaggregated carbon inventory for the conterminous United States}
\author{Neil Best, Joshua Elliott\thanks{nbest@ci.uchicago.edu, jelliott@ci.uchicago.edu}}
\date{}
\hypersetup{
  pdfkeywords={woody biomass, land cover, protected areas},
  pdfsubject={Where does the description appear?},
  pdfcreator={Emacs Org-mode version 7.9.4}}

\begin{document}

\maketitle

% Org-mode is exporting headings to 2 levels.




This technical paper describes and reveals the process of creating a
disaggregated carbon inventory dataset based on three base datasets.
The carbon density distribution for above-ground live dry woody
biomass is taken from the National Biomass and Carbon Dataset (NBCD)
from the Woods Hole Research Center (WHRC).  We use the National Land
Cover Dataset (NLCD) of 2001 to assign each carbon density reading to
a cover class.  The Protected Areas Database for the United States
(PAD-US) is used to assign distinction within those land cover classes
based on legal protection status.  This tabulation of the overall
woody biomass inventory is summarized at three levels of
disggregation: state, county, and 5 arc-minute ($5'$) grid cells.  The
disaggregation by political unit is useful for policy analysis and the
gridded representation is a key input for continental-scale land use
conversion analyses that aim to capture the complete carbon cycle
implications of converting forested land to biofuel crops, for
example.
\tableofcontents
\listoftables
\listoffigures

\section{Introduction}
\label{sec-1}
\subsection{\textbf{TODO} [exact name of biofuels program]}
\label{sec-1-1}


some prose about the biofuels program?
\subsection{Motivation}
\label{sec-1-2}


To apply the information contained in the National Biomass and Carbon
Dataset (NBCD) to an analysis of the land use change implications of
expanding biofuels production it is necessary to add qualitative depth
to the data.  There are components of the total above-ground live dry
biomass (ALDB) that should be treated differently when modelling
economic decisions about land use change and land management.  Rather
than simply adding up the ALDB for each geographical unit in a given
set of spatial boundaries that total biomass can be broken out into
categories by overlaying information from other sources.  To add
information about the physical characteristics of the biomass
indicated at a given location and its cultural or ecological context
we use the National Land Cover Database (NLCD).  Similarly to
distignuish areas based on their legal protection status we bring in
the Protected Areas Database of the United States (PAD-US). By
assigning attributes from NLCD and PAD-US to the corresponding biomass
inventory indicated by a given pixel in the NBCD data we are able to
disaggregate the total biomass into categories given by the Cartesian
product of the individual datasets' categories. More detail on this in
section ______.
\subsection{Analysis Framework}
\label{sec-1-3}


A secondary goal of this work is to demonstrate a framework for
reproducible research.  Unless you are browsing the source code
repository of this project you are reading a document that is the
result of an automated process that downloads the base data, performs
some necessary transformations of those datasets, performs the
computations from top to bottom, updates the figures, tables, and
maps, and exports the complete document to a PDF file.  The particular
framework chosen for this work uses the Emacs editor, Org-mode for
outlining and literate programming, the Geospatial Data Abstraction
Library (GDAL) for manipulation of vector and raster datasets, GRASS
for analysis and tabulation of GIS data, and R for aggregation and
formatting of the data. 
\subsection{Summary of Methodology}
\label{sec-1-4}


The initial goal of this work was to assign protection status to areas
of the various land cover classes.  Because PAD-US has coverage
everywhere that NLCD does we extended the analysis to include Alaska,
Hawaii, and Puerto Rico.  However, at present NBCD only provides data
for the conterminous United States (cUSA) so the ultimate result of
this process only covers that region.  The tabulation of protected
areas by cover clasee for the outlying areas is provided for
completeness leaving open the possibility that more data might become
available in the future.

To satisfy various modelling and reporting requirements we have
adopted three modes of disaggregation of the woody biomass: state,
county, and 5 arc-minute ($5'$, one twelfth of a degree) grid cells.
Clearly the first two are dependent and nested -- the total biomass in
a state is the sum of the biomass of its counties.  Apart from the
political boundaries it is common practice to subdivide the globe or
regions thereof into units based on geographic coordinates,
i.e. longitude and latitude.  At $5'$ resolution a grid cell is
roughly 10 km on a side at the equator.  The variation in area of
these grid cells must be accounted for in any comparison among or
summation across grid cells, but this complication is accepted as a
trade-off against working in familiar map coordinates.

In all cases we chose to reproject the units of disaggregation, being
the states, counties, and grid cells, into the native representation
of the input raster data (NBCD and NLCD), which happen to be in
regional Albers equal-area projections chosen to minimize area
distortions for their respective extents: cUSA, Alaska, Hawaii, and
Purto Rico.  The political boundaries and protection status
delineations are provided in vector format making them less
computationally intensive to reproject than their raster counterparts.
After reprojection we convert the political and protection status
boundaries into their raster representation at the same resolution as
the NBCD and NLCD data, thereby assigning additional qualitative
categories to each NBCD/NLCD pixel.  From a programming perspective it
was expedient to reproject the grid cells in raster space but they
just as well could have been defined as vector entities as well.
Therefore assigning NBCD pixels to their respective $5'$ grid cells
was perhaps more expensinve in terms of processing time to perform but
simpler from a conceptual and implementation perspective.  By bringing
our reporting units into the coordinate space of the NBCD data we
guarantee that the sum total of all given biomass will be conserved.
\section{Underlying datasets}
\label{sec-2}


This section describes the input datasets that go into this analysis
in greater detail than what was offerd in the introduction.
Conceptually the analysis employs land use / land cover information
from NLCD, legal protection status from PAD-US, and political
boundaries from the US Census to disaggregate undifferentiated
estimates of woody biomass, i.e. ALDB, which is assumed to be 50\%
carbon by mass, taken from NBCD.  The analysis also employs another
spatial divisions of disaggregation, the $5'$ grid cells, which are
also defined here for completeness.
\subsection{\textbf{TODO} Describe NBCD ALDB and BAWH}
\label{sec-2-1}
\subsection{\textbf{TODO} cite NBCD web site and in-review paper}
\label{sec-2-2}


Kellndorfer, J., Walker, W., LaPoint, E., Bishop, J., Cormier, T.,
Fiske, G., Hoppus, M., Kirsch, K., and Westfall, J. 2012. NACP
Aboveground Biomass and Carbon Baseline Data (NBCD 2000),
U.S.A., 2000. Data set. Available on-line [http://daac.ornl.gov] from
ORNL DAAC, Oak Ridge, Tennessee,
U.S.A. \href{http://dx.doi.org/10.3334/ORNLDAAC/1081}{http://dx.doi.org/10.3334/ORNLDAAC/1081} or at
\href{http://www.whrc.org/mapping/nbcd/}{http://www.whrc.org/mapping/nbcd/}

Kellndorfer, J.M., Walker, W.S., Kirsch, K.M., Fiske, G., Bishop, J.,
LaPoint, E., Hoppus, M, \& Westfall, J. The National Biomass and Carbon
data set 2000 (NBCD 2000): A high resolution baseline for the
conterminous U.S. Remote Sensing of Environment, Submitted.
\subsection{NLCD land use/land cover and canopy density}
\label{sec-2-3}
\subsection{Dependency of NBCD on NLCD}
\label{sec-2-4}


One important factor that influenced our operational choices in
carrying out the analysis is the dependence between NLCD and NBCD.
NBCD's ALDB densities are only calculated where NLCD's canopy density
is non-zero and WHRC's radar-derived basal area weighted height
(BAWH) is also non-zero.  Because NBCD is modelled based on the first
version of the 2001 NLCD and that is the only version for which
canopy density is provided it is meaningless to disaggregate ALDB
based on subsequent versions of NLCD.  The classification accuracy of
NLCD 2001 version 1 is described in _____.  
\section{Analysis Procedure}
\label{sec-3}


Conceptually our method reduces to a brute-force tabulation exercise.
We are simply constructing classifications of the following forms:

\begin{itemize}
\item grid ID, land use/cover, protection status
\item state, county, land use/cover, protection status
\end{itemize}

By tabulating NBCD pixels on these bases we are able sum up the ALDB
densities for a given set of identifiers in the above schema.  For
this method to work it is necessary to convert all of the variables
into rasters of the same projection, extent, and resolution so that
there is a one-to-one relationship between pixels among the variables.
For this analysis we have chosen to work in the Albers equal-area
projection used by NBCD and NLCD to avoid performing any
transformation on them.  PAD-US and the political boundaries are
provided in vector form so they are reprojected and converted to
rasters.  Similarly the $5'$ grid cells that are regular in
geographic coordinates are reprojected into irregular areas in the
target projection so that each NBCD/NLCD pixel is assigned a grid
cell ID.
\subsection{Reproject the $5'$ grid cells}
\label{sec-3-1}


For convenience we adopt a scheme for identifying the $5'$ grid cells
using sequential integers rather than decimal coordinates.  By
stipulating that any grid we wish to consider in our models is either
a global grid or a subset thereof the only parameter that must be
tracked is the resolution.  Cell number one has its upper left corner
at 180 degrees west longitude and 90 degrees north latitude (-180,90)
by convention.  From there the cell numbers increase to the east and
then resume at the western edge when eastern edge is reached.  \texttt{R}'s
\texttt{raster} package provides a convenient abstraction for specifying and
instantiating such simply formulated rasters.  The following code
performs these steps:

\begin{enumerate}
\item Create a global raster in geogrpahic coordinates by default.
\item Set the resolution to $5'$.
\item Assign values to its cells from a sequence that runs from one to
   the number of cells.
\item Write the data to a file.
\end{enumerate}


\begin{verbatim}
world <- raster()
res( world) <- 5/60
world[ ] <-
  1:ncell( world)
world <-
  writeRaster(
    world, "data/grid5minWorld.tif",
    datatype= "INT4U",
    overwrite= overwriteRasters)
\end{verbatim}

The cUSA study area covers only a subset of these cells.  Using the
\texttt{gdalwarp} utility from GDAL we can specify a target projection,
extent, and resolution that match NBCD/NLCD.


\begin{verbatim}
cusaProj='+proj=aea +lat_1=29.5 +lat_2=45.5 +lat_0=23 cusaProj'
cusaProj+='+lon_0=-96 +x_0=0 +y_0=0 +ellps=GRS80 +units=m +no_defs'

gdalwarp -overwrite -of HFA \
    -t_srs ${cusaProj} \
    -te -2493045 177285 2342655 3310005 \
    -tr 30 30 -co "COMPRESSED=YES" \
    data/grid5minWorld.tif data/grid5minAeaCUSA.img
\end{verbatim}
\subsection{Download and prepare inputs}
\label{sec-3-2}
\subsubsection{NLCD}
\label{sec-3-2-1}


The first version of the 2001 NLCD is available as a single raster
file so no special preparation is required.
\subsubsection{NBCD}
\label{sec-3-2-2}


The second version of the NBCD is divided according to ``mapping zones''
and so must be patched together to form a continuous coverage.  We do
this using GDAL's \texttt{gdalbuildvrt} to build a virtual raster from the
collection of downloaded mapping zone tiles.  The caveat which comes
with this approach is that the individual NBCD rasters include a small
buffer outside the boundary of the respective mapping zone which means
that there will be one or more NBCD rasters that contain estimates for
pixels on either side of the mapping zone boundaries.  Because the
model that produces the NBCD estimates has a stochastic component
estimates for a given pixel within these overlapping buffer zones are
not constrained to have the same value.  From \href{http://gdal.org/gdalbuildvrt.html}{the \texttt{gdalbuildvrt} page on gdal.org}:

\begin{quote}
If there is some amount of spatial overlapping between files, the
order may depend on the order they are inserted in the VRT file, but
this behaviour should not be relied on.
\end{quote}

The best action we could take to maintain reproducibility in this
situation is to ensure that the arguments to the \texttt{gdalbuildvrt}
utility, the names of the NBCD tiles, are given to in lexicographic
order.
\begin{itemize}

\item \textbf{TODO} either copy or include code from NBCD submodule
\label{sec-3-2-2-1}%


\end{itemize} % ends low level
\subsubsection{PAD-US}
\label{sec-3-2-3}
\subsubsection{State and county boundaries}
\label{sec-3-2-4}
\begin{itemize}

\item \textbf{TODO} check if \texttt{raster} has improved tabulation performance.
\label{sec-3-3-1}%


\end{itemize} % ends low level
\subsection{Load the data into GRASS}
\label{sec-3-3}


Due to the volume of data involved in computations at continental
extents and 30m resolution processing performance becomes a critical
factor.  At the time of this writing the performance of \texttt{GRASS}'s
\texttt{r.stats} utility far outstrips that of Hijman \& van Etten's \texttt{raster}
package for \texttt{R} in computing pixel count tabulations.  Therefore it is
worth the trouble to populate as \texttt{GRASS} database and convolve our
input data sets there.
\subsubsection{start GRASS}
\label{sec-3-3-1}



\begin{verbatim}
cd ~/Desktop/pad-us_nlcd
export GISRC=./.grassrc6
grass64 -text data/grass/cUSA/PERMANENT
\end{verbatim}
\begin{itemize}

\item \textbf{TODO} check that GRASS processing works out of the box\\
\label{sec-3-3-1-1}%
Make sure that the bare GRASS database in the Git repo is enough to
start the project cleanly.


\end{itemize} % ends low level
\subsubsection{Load the NBCD data}
\label{sec-3-3-2}



\begin{verbatim}
# g.gisenv set=LOCATION_NAME=cUSA
# g.gisenv set=MAPSET=PERMANENT
eval $(g.gisenv)

# export GRASS_MESSAGE_FORMAT=plain 
r.in.gdal --overwrite -e \
  input=nbcd/data/nbcdAldb.vrt output=nbcdAldb
r.in.gdal --overwrite -e \
  input=nbcd/data/nbcdBawh.vrt output=nbcdBawh
\end{verbatim}
\subsubsection{load the $5'$ grid cell IDs}
\label{sec-3-3-3}



\begin{verbatim}
r.in.gdal in=data/grid5minAeaCUSA.vrt out=grid_5min
\end{verbatim}
\subsubsection{laod the PAD-US data}
\label{sec-3-3-4}



\begin{verbatim}
r.in.gdal input=pad-us/data/PADUS1_2_cUSA_GAP.tif output=gap
\end{verbatim}
\subsubsection{Load the NLCD data}
\label{sec-3-3-5}



\begin{verbatim}
r.in.gdal input=nlcd/pr_landcover_wimperv_10-28-08_se5.img output=Nlcd01v1PR
r.in.gdal --overwrite -e \
  input=nlcd/data/nlcd2001_canopy_mosaic_1-29-08/nlcd_canopy_mosaic_1-29-08.img \
  output=canopy
\end{verbatim}
\subsubsection{Load the state and county rasters}
\label{sec-3-3-6}



\begin{verbatim}
r.in.gdal --overwrite -e \
  input=data/cusaStatesAea.img output=states
r.in.gdal --overwrite -e \
  input=data/cusaCountiesAea.img output=counties
\end{verbatim}
\subsection{Replace null values in NBCD with zeroes}
\label{sec-3-4}



\begin{verbatim}
g.region rast=Nlcd01v1
r.mask -o input=Nlcd01v1 maskcats="1 thru 95"

r.mapcalc nbcdMask='eval( nbcd=canopy > 0 && nbcdBawh > 0, if( isnull( nbcd), 0, nbcd))'

# echo grid_5min,Nlcd01v1,gap,aldb,n > data/statsNbcdNlcd01v1Grid5min.csv && \
# r.stats -Nc input=grid_5min,Nlcd01v1,gap,nbcdAldb fs=, >> data/statsNbcdNlcd01v1Grid5min.csv 2> data/statsNbcdNlcd01v1Grid5min.err &

echo state,county,Nlcd01v1,gap,nbcd,aldb,n > data/statsNbcdNlcd01v1County.csv \
    && r.stats -Nc input=states,counties,Nlcd01v1,gap,nbcdMask,nbcdAldb fs=, \
    >> data/statsNbcdNlcd01v1County.csv \
    2> data/statsNbcdNlcd01v1County.err &
\end{verbatim}
\subsection{Aggregate the statistics}
\label{sec-3-5}
\subsubsection{by 5' grid cells}
\label{sec-3-5-1}



\begin{verbatim}
stats <-
  read.csv( "statsNbcd.csv",
           na.strings= "*",
           col.names= c( "cell", "nlcd", "gap", "nbcd", "n"),
           colClasses= c("numeric", "factor", "factor", "numeric"))

stats <-
  within(
    stats,
    { levels( gap) <- c( levels( gap), "0")
      gap[ is.na( gap)] <- "0"
      gap <- combine_factor( gap, c(0,1,1,1,0))
      levels( gap) <- c( "no", "yes")
      nbcd[ is.na( nbcd)] <- 0
    })

dt <- data.table( stats)
setkey( dt, cell, nlcd, gap)

wm <- dt[, list( wm= weighted.mean( nbcd, n)), by= "cell,nlcd,gap"]

wmCt <-
  cast(
    data= wm,
    formula= cell ~ gap + nlcd,
    ## fun.aggregate= sum,
    ## margins= "grand_col",
    value= "wm" )

write.csv(
  format.df(
    wmCt,
    cdec= c( 0, rep( 1, ncol( wmCt) - 1)),
    numeric.dollar= FALSE,
    na.blank= TRUE),
  row.names= FALSE,
  file= "nbcdFiaAldb.csv",
  quote= FALSE)

zip( "pad-us_nlcd_nbcd.zip", "fracscUSA.csv")
zip( "pad-us_nlcd_nbcd.zip", "nbcdFiaAldb.csv")
\end{verbatim}
  
\subsubsection{load r.stats output}
\label{sec-3-5-2}



\begin{verbatim}

library( reshape)
library( Hmisc)
library( data.table)
library( stringr)
library( ggplot2)
library( foreign)

stateAttrs <-
  read.dbf( "shp/tl_2010_us_state10.dbf")
stateNames <-
  data.table( stateAttrs[, c( "STATEFP10", "STUSPS10", "NAME10")])
setnames(
  stateNames,
  names( stateNames),
  c( "state", "usps", "name"))
setkey( stateNames, state)

keycols <- c(
  "state", "county", "nlcd", 
  "gap",  "nbcd", "aldb")

rawCountyStats <-
  read.csv(
    "data/statsNbcdNlcd01v1County.csv",
    na.strings= "*",
    header= TRUE,
    col.names= c( keycols, "n"),
    colClasses= c(
      "character", "character", "character",
      "integer", "integer", "integer", "integer"))

rawCountyStats <- data.table( rawCountyStats)
setkeyv( rawCountyStats, keycols)

{
  rawCountyStats[ is.na(  state),  state := "0"]
  rawCountyStats[ is.na( county), county := "0"]
  rawCountyStats[,  state :=
                 str_pad(
                   as.character(  state),
                   2, pad= "0")]
  rawCountyStats[, county :=
                 str_pad(
                   as.character( county),
                   3, pad= "0")]
  rawCountyStats[ is.na( gap), gap := 0L]
  rawCountyStats[ gap == 4L, gap := 0L]
  rawCountyStats[ gap != 0L, gap := 1L]
  rawCountyStats[ is.na( aldb), aldb := 0L]
  rawCountyStats[, gap := as.logical( gap)]
  rawCountyStats[, nbcd := as.logical( nbcd)]
  rawCountyStats[ !nbcd, aldb := NA]
}

setkeyv( rawCountyStats, keycols)
\end{verbatim}
\subsubsection{collapse the records for GAP values that are no longer differentiated}
\label{sec-3-5-3}




\begin{verbatim}

rawCountyStats <-
  rawCountyStats[, list( n= sum( n)),
                 keyby= keycols ]

rawStateStats <- 
  rawCountyStats[, list( n= sum( n)),
                 keyby= keycols[ -2] ]


## same thing in functional form

## rawStateStats <- `[`(
##   x= rawCountyStats,
##   j= list( n= sum( n)),
##   keyby= keycols[ -2])
\end{verbatim}

Show the low-density forests that we intend to backfill.



\begin{verbatim}

qplot(
  data= rawCountyStats[ nlcd %in% as.character( c(41:43, 90))],
  x= aldb,
  weight= n,
  binwidth= 5,
  xlim= c( 0, 200),
  geom= "histogram", 
  group= gap,
  fill= gap,
  position= "dodge",
  ## position= "stack",
  facets= nlcd ~ .)
\end{verbatim}

\includegraphics[width=.9\linewidth]{pad-us_nlcd/images/forestHist.png}



\begin{verbatim}

qplot(
  data= rawCountyStats[ nlcd %in% as.character( c(41:43, 90))],
  x= aldb,
  weight= n,
  binwidth= 1,
  xlim= c( 0, 20),
  geom= "histogram", 
  group= gap,
  fill= gap,
  position= "dodge",
  ## position= "stack",
  facets= nlcd ~ .)
\end{verbatim}

\includegraphics[width=.9\linewidth]{pad-us_nlcd/images/forestHistDetail.png}
\subsubsection{repeat for states}
\label{sec-3-5-4}
\begin{itemize}

\item calculate the average densities for original and modified pixel counts
\label{sec-3-5-4-1}%

\begin{itemize}

\item \textbf{TODO} Figure out how to do self-join in functional form\\
\label{sec-3-5-4-1-1}%
\begin{verbatim}

statsNbcdState <- `[`(
  x= rawStateStats,
  j= list(
    aldb= weighted.mean( aldb, n),
    n= sum( n),
    ha= sum(n) * 30^2 / 10^4),
  keyby= "state,nlcd,gap,nbcd")

stateAreas <- `[`(
  x= statsNbcdState ,
  j= list( totHa= sum(ha)),
  keyby= "state")

statsNbcdState <-
  statsNbcdState[ stateAreas][, frac:= ha / totHa][, totHa := NULL]


## statsNbcdState <- `[`(
##   x= `[`(
##     x= statsNbcdState,
##     j= list( stateAreas)) ,
##   j= frac:= ha / totHa)

## `[`( x= statsNbcdState, j= stateAreas)

## `[`( x= statsNbcdState,
##     j= list(
##       names( statsNbcdState),
##       frac= ha / sum( .SD[, ha])),
##     keyby= "state")
\end{verbatim}


To backfill those NLCD/GAP combinations we must calculate national averages.


\begin{verbatim}
nbcdMean <-
  statsNbcdState[ n != n2][, list(
                    aldb= weighted.mean( aldb2, n, na.rm= TRUE)),
                    keyby= "nlcd,gap"]

ascii(
  cast(
    nbcdMean,
    nlcd ~ gap,
    value= "aldb"),
  digits = 1,
  include.rownames= FALSE)
\end{verbatim}


\begin{center}
\begin{tabular}{rrr}
 nlcd  &  FALSE  &   TRUE  \\
\hline
   41  &  111.4  &  119.5  \\
   42  &  103.7  &  120.5  \\
   43  &  110.2  &  134.0  \\
   90  &   96.4  &   91.1  \\
\end{tabular}
\end{center}








\end{itemize} % ends low level

\item calculate the grand total\\
\label{sec-3-5-4-2}%
\begin{verbatim}
statsNbcdState[ I( nbcd), list( Gt= sum( aldb * ha) / 10^9)]
\end{verbatim}



\item write out the state data in long form\\
\label{sec-3-5-4-3}%
\begin{verbatim}

write.csv(
  statsNbcdState[, list( state,nlcd,gap,nbcd,aldb,n,ha,frac)],
  row.names= FALSE,
  file= "pad-us_nlcd/nbcdStateSerial.csv",
  quote= FALSE)

zip( "pad-us_nlcd.zip", "pad-us_nlcd/nbcdStateSerial.csv")
\end{verbatim}




\item cross-tabulate the state data using backfilled densities\\
\label{sec-3-5-4-4}%
\begin{verbatim}

nbcdStateAldb <- 
  data.table(
    cast(
      data= statsNbcdState,
      subset= nbcd,
      formula= state ~ gap + nlcd,
      value= "aldb"),
    key= "state")

setnames(
  nbcdStateAldb,
  colnames( nbcdStateAldb),
  str_replace(
    str_replace(
      colnames(nbcdStateAldb),
      "TRUE_", "yes"),
    "FALSE_", "no"))


nbcdStateAldbMeans <- 
  data.table(
    cast(
      data=
      statsNbcdState[
        I( nbcd),
        list( aldbAve= weighted.mean( aldb, ha)),
        by= c( "state", "gap")],
      formula= state ~ gap,
      value= "aldbAve"),
    key= "state")

setnames(
  nbcdStateAldbMeans,
  c( "FALSE", "TRUE"),
  c( "noAll", "yesAll"))

nbcdStateGapFrac <-
  data.table(
    cast(
      data= statsNbcdState,
      formula= state ~ gap,
      value= "frac",
      fun.aggregate= sum,
      na.rm = TRUE),
    key= "state")

setnames(
  nbcdStateGapFrac,
  c( "FALSE", "TRUE"),
  c( "noAllFr", "yesAllFr"))

nbcdStateGapHa <-
  data.table(
    cast(
      data= statsNbcdState,
      formula= state ~ gap,
      value= "ha",
      fun.aggregate= sum,
      na.rm= TRUE),
    key= "state")

setnames(
  nbcdStateGapHa,
  c( "FALSE", "TRUE"),
  c( "noAllHa", "yesAllHa"))

nbcdStateGapHa <-
  data.table(
    cast(
      data= statsNbcdState[, list( ha=sum( ha)),
        keyby= "state,gap,nbcd"],
      formula= state ~ gap + nbcd,
      value= "ha"),
    key= "state")

setnames(
  nbcdStateGapHa,
  c( "FALSE_FALSE", "FALSE_TRUE", "TRUE_FALSE", "TRUE_TRUE"),
  c( "noAllHaN", "noAllHa", "yesAllHaN", "yesAllHa"))

nbcdStateFrac <- 
  data.table(
    cast(
      data= statsNbcdState,
      subset= nbcd,
      formula= state ~ gap + nlcd,
      value= "frac"),
    key= "state")

setnames(
  nbcdStateFrac,
  colnames( nbcdStateFrac)[ -1],
  paste(
    str_replace(
      str_replace(
        colnames( nbcdStateFrac)[ -1],
        "TRUE_", "yes"),
      "FALSE_", "no"),
    "Fr", sep= ""))

nbcdNullStateFrac <- 
  data.table(
    cast(
      data= statsNbcdState,
      subset= !nbcd,
      formula= state ~ gap + nlcd,
      value= "frac"),
    key= "state")

setnames(
  nbcdNullStateFrac,
  colnames( nbcdNullStateFrac)[ -1],
  paste(
    str_replace(
      str_replace(
        colnames( nbcdNullStateFrac)[ -1],
        "TRUE_", "yes"),
      "FALSE_", "no"),
    "FrN", sep= ""))

nbcdStateHa <- 
  data.table(
    cast(
      data= statsNbcdState,
      subset= nbcd,
      formula= state ~ gap + nlcd,
      value= "ha"),
    key= "state")

setnames(
  nbcdStateHa,
  colnames( nbcdStateHa)[ -1],
  paste(
    str_replace(
      str_replace(
        colnames( nbcdStateHa)[ -1],
        "TRUE_", "yes"),
      "FALSE_", "no"),
    "Ha", sep= ""))

nbcdNullStateHa <- 
  data.table(
    cast(
      data= statsNbcdState,
      subset= !nbcd,
      formula= state ~ gap + nlcd,
      value= "ha"),
    key= "state")

setnames(
  nbcdNullStateHa,
  colnames( nbcdNullStateHa)[ -1],
  paste(
    str_replace(
      str_replace(
        colnames( nbcdNullStateHa)[ -1],
        "TRUE_", "yes"),
      "FALSE_", "no"),
    "HaN", sep= ""))

nbcdState <-
  nbcdStateAldb[ nbcdStateAldbMeans]
nbcdState <-
  nbcdState[ nbcdStateGapFrac][ nbcdStateGapHa]
nbcdState <-
  nbcdState[ nbcdStateFrac][ nbcdStateHa]
nbcdState <-
  nbcdState[ nbcdNullStateFrac][ nbcdNullStateHa]

setnames(
  nbcdState,
  "state", "fips")

setcolorder(
  nbcdState,
  order( colnames( nbcdState)))

nbcdStateChar <-
  str_trim(
    format.df(
      nbcdState,
      cdec= sapply(
        colnames( nbcdState),
        function( x) {
          ifelse(
            x == "fips", 0,
            ifelse(
              str_detect( x, "HaN?$"), 1,
              ifelse(
                str_detect( x, "FrN?$"), 3,
                1)))
        }),
      numeric.dollar= FALSE,
      na.blank= TRUE))

write.csv(
  nbcdStateChar,
  row.names= FALSE,
  file= "pad-us_nlcd/nbcdState.csv",
  quote= FALSE)

zip( "pad-us_nlcd.zip", "pad-us_nlcd/nbcdState.csv")

options(useFancyQuotes = FALSE)
cat(
  sapply(
    colnames( nbcdState),
    function( x) {
      dQuote(
        ifelse(
          x == "fips", "String(2)",
          ifelse(
            str_detect( x, "HaN?$"),
            "Real(10.1)",
            ifelse(
              str_detect( x, "FrN?$"),
              "Real(5.3)",
              "Real(5.1)"))))
    }),
  sep= ",",
  file= "pad-us_nlcd/nbcdState.csvt")
options(useFancyQuotes = TRUE)

ogr2ogr <-
  paste(
    "ogr2ogr -overwrite -progress -sql",
    sprintf(
      "\"select %s from cusaStatesAea a",
      paste( colnames( nbcdState), collapse= ",")),
    "left join 'pad-us_nlcd/nbcdState.csv'.nbcdState b",
    "on a.GEOID10 = b.fips\"",
    "pad-us_nlcd/nbcdState.shp shp/cusaStatesAea.shp")

system( ogr2ogr)

zip(
  "pad-us_nlcd.zip",
  list.files(
    path= "pad-us_nlcd",
    pattern= "^nbcdState\\.[a-z]{3}",
    full.names= TRUE))
\end{verbatim}




\end{itemize} % ends low level
\subsubsection{repeat for counties}
\label{sec-3-5-5}



\begin{verbatim}
## library( reshape)
## library( Hmisc)
## library( data.table)
## library( stringr)

statsNbcdCounty <-
  rawCountyStats[, list(
    aldb= weighted.mean( aldb, n),
    n= sum( n),
    ha= sum(n) * 30^2 / 10^4),
                 keyby= "state,county,nlcd,gap,nbcd"]
countyAreas <-
  statsNbcdCounty[, list( totHa= sum(ha)),
                  keyby= c( "state", "county")]
statsNbcdCounty <-
  statsNbcdCounty[ countyAreas][, frac:=ha/totHa]

## statsNbcdCounty <-
##   merge(
##     statsNbcdCounty,
##     statsNbcdState[, list( state, nlcd, gap, aldb2, n2)],
##     by= c( "state", "nlcd", "gap"),
##     all.x= TRUE)

## statsNbcdCounty[, fill := as.character(NA)]

## statsNbcdCounty[ n2.x == 0 & n2.y == 0, fill := "cUSA"]
## statsNbcdCounty[ n2.x == 0 & n2.y != 0, fill := "state"]

## statsNbcdCounty[ is.na( aldb2.x), aldb2.x := aldb2.y]
## statsNbcdCounty[, aldb2.y := NULL]
## statsNbcdCounty[,    n2.y := NULL]
## statsNbcdCounty[,   totHa := NULL]

## setkey( statsNbcdCounty, state, county, nlcd, gap)
## setnames(
##   statsNbcdCounty,
##   c( "aldb2.x", "n2.x"),
##   c( "aldb2",   "n2"))

## setcolorder(
##   statsNbcdCounty,
##   c( "state", "county", "nlcd", "gap",
##     "aldb", "aldb2", "n", "n2", "ha", "ha2",
##     "frac", "fill"))

## test
## any( abs( statsNbcdCounty[, list( frac= sum(frac)), by= "state,county"][, frac] - 1) > 0.001)
\end{verbatim}
\begin{itemize}

\item check for cases where NBCD makes no prediction\\
\label{sec-3-5-5-1}%
NBCD only predicts biomass density where it has sufficient canopy
density and basal area-weighted height to do so.  We can presume that
the ALDB density for a given state/county//nlcd/gap combination outside of the
NBCD prediction is something less than its modeled counterpart.  Any
heuristic that attempts to plug in a density value not predicted by
NBCD will fail when the NBCD prediction area was zero for a given
state/nlcd/gap combination.


\begin{verbatim}
countyAreaCheck <- 
  data.table( cast(
    statsNbcdCounty,
    state + county + nlcd + gap ~ nbcd,
    value= "n"))

countyAreaCheck <-
  countyAreaCheck[ `FALSE` > 0 & `TRUE` == 0]
setkey( countyAreaCheck, state, county, nlcd, gap)


statsNbcdCountyTrue <- statsNbcdCounty[ I( nbcd)][, nbcd := NULL]
setkey( statsNbcdCountyTrue, state, county, nlcd, gap)

statsNbcdCountyFalse <- statsNbcdCounty[ !nbcd][, nbcd := NULL]
setkey( statsNbcdCountyFalse, state, county, nlcd, gap)

statsNbcdCountyTrue[ statsNbcdCountyFalse][ is.na( aldb)]

cast(
  data.frame(
    statsNbcdCounty[ !nbcd]),
    ## statsNbcdCountyTrue[ statsNbcdCountyFalse][ is.na( aldb)]),
  nlcd ~ gap,
  value= "ha",
  fun= sum,
  margins= TRUE)


aldbIgnoringGap <-
  statsNbcdCounty[ I( nbcd),
                  list(
                    aldb= weighted.mean(
                      aldb, n, na.rm= TRUE)),
                  keyby= "state,county,nlcd"]


cast(
  data.frame(
    ),
  nlcd ~ gap,
  value= "ha.1",
  fun= sum,
  margins= TRUE)


cast(
  data.frame(
    statsNbcdCounty),
  nlcd ~ gap,
  value= "ha",
  fun= sum,
  margins= TRUE)


statsNbcdCounty[ !nbcd, ha,
  keyby= "state,county,nlcd,gap"][
    statsNbcdCounty[, ha,
      keyby= "state,county,nlcd,gap"],
    nomatch=0][,
      list( haFalse= sum( ha),
           haTotal= sum( ha.1)), by="nlcd,gap"]

notModeledByNbcd <- 
  statsNbcdCounty[ !nbcd, ha, keyby= "state,county,nlcd,gap"][
    statsNbcdCounty[ , list( totHa= sum(ha)), keyby= "state,county,nlcd,gap"], nomatch=0][,
      list( ha= sum(ha), totHa= sum( totHa)), keyby= "nlcd,gap"]


notModeledPcts <- 
  data.frame(
    notModeledByNbcd[,
      pct := ha / totHa * 100])

within( notModeledPcts, {
  nlcd <- factor(nlcd, levels= names( nlcdCovers), labels= nlcdCovers)
}

ascii(
  cast(
    within(
      notModeledPcts,
      nlcd <- factor(
        nlcd,
        levels= names( nlcdCovers),
        labels= nlcdCovers)),
    nlcd ~ gap, value= "pct"),
  digits=3,
  include.rownames= FALSE,
  colnames= c( "NLCD", "unprotected", "protected"),
  align= c( "l", "r", "r"))
\end{verbatim}



\item \textbf{TODO} Make a table of these results\\
\label{sec-3-5-5-2}%
\begin{center}
\begin{tabular}{rrrr}
 state  &  FALSE  &  TRUE  &  (all)  \\
\hline
    00  &      0  &     0  &      0  \\
    01  &    829  &    49  &    878  \\
    04  &     29  &   268  &    297  \\
    05  &    592  &   156  &    749  \\
    06  &    823  &  1287  &   2111  \\
    08  &    152  &   534  &    687  \\
    09  &    123  &    21  &    144  \\
    10  &     17  &     5  &     22  \\
    11  &      1  &     0  &      1  \\
    12  &    430  &   179  &    609  \\
    13  &    857  &    97  &    954  \\
    16  &    128  &   666  &    795  \\
    17  &    221  &    32  &    254  \\
    18  &    258  &    31  &    289  \\
    19  &    105  &     6  &    112  \\
    20  &     70  &     3  &     72  \\
    21  &    618  &    64  &    682  \\
    22  &    504  &    68  &    572  \\
    23  &    629  &   141  &    770  \\
    24  &    132  &    31  &    163  \\
    25  &    162  &    62  &    224  \\
    26  &    521  &   252  &    773  \\
    27  &    260  &   254  &    514  \\
    28  &    659  &    69  &    728  \\
    29  &    534  &    93  &    627  \\
    30  &    186  &   672  &    858  \\
    31  &     37  &     4  &     41  \\
    32  &     14  &   126  &    141  \\
    33  &    205  &    95  &    300  \\
    34  &     97  &    55  &    152  \\
    35  &     95  &   233  &    327  \\
    36  &    860  &   297  &   1157  \\
    37  &    812  &   154  &    965  \\
    38  &     20  &     7  &     27  \\
    39  &    444  &    47  &    492  \\
    40  &    285  &    33  &    318  \\
    41  &    606  &  1229  &   1835  \\
    42  &    828  &   270  &   1098  \\
    44  &     18  &     6  &     24  \\
    45  &    447  &    56  &    503  \\
    46  &     24  &    30  &     54  \\
    47  &    629  &   102  &    730  \\
    48  &    653  &    66  &    719  \\
    49  &     68  &   242  &    310  \\
    50  &    219  &    60  &    279  \\
    51  &    765  &   155  &    920  \\
    53  &    582  &  1007  &   1589  \\
    54  &    666  &   110  &    776  \\
    55  &    471  &    95  &    566  \\
    56  &     48  &   287  &    335  \\
 (all)  &  17736  &  9807  &  27543  \\
\end{tabular}
\end{center}





\item calculate the grand total\\
\label{sec-3-5-5-3}%
\begin{verbatim}
statsNbcdCounty[ I( nbcd), list( Gt= sum( aldb * ha) / 10^9)]
\end{verbatim}



\item write out the results in long form\\
\label{sec-3-5-5-4}%
\begin{verbatim}

write.csv(
  statsNbcdCounty[, list( state, county, nlcd, gap, nbcd, aldb, ha, frac)],
  row.names= FALSE,
  file= "pad-us_nlcd/nbcdCountySerial.csv",
  quote= FALSE)

zip( "pad-us_nlcd.zip", "pad-us_nlcd/nbcdCountySerial.csv")
\end{verbatim}


\item cross-tabulate the county data\\
\label{sec-3-5-5-5}%
\begin{verbatim}
nbcdCountyAldb <- 
  data.table(
    cast(
      data= statsNbcdCounty,
      subset= nbcd,
      formula= state + county ~ gap + nlcd,
      value= "aldb"),
    key= "state,county")

setnames(
  nbcdCountyAldb,
  colnames( nbcdCountyAldb),
  str_replace(
    str_replace(
      colnames( nbcdCountyAldb),
      "TRUE_", "yes"),
    "FALSE_", "no"))

nbcdCountyAldbMeans <- 
  data.table(
    cast(
      data=
      statsNbcdCounty[
        I( nbcd),
        list( aldbAve= weighted.mean( aldb, ha, na.rm= TRUE)),
        by= c( "state", "county", "gap")],
      formula= state + county ~ gap,
      value= "aldbAve"),
    key= "state,county")

setnames(
  nbcdCountyAldbMeans,
  c( "FALSE", "TRUE"),
  c( "noAll", "yesAll"))

nbcdCountyGapFrac <-
  data.table(
    cast(
      data= statsNbcdCounty,
      formula= state + county ~ gap,
      value= "frac",
      fun.aggregate= sum,
      na.rm= TRUE),
    key= "state,county")

setnames(
  nbcdCountyGapFrac,
  c( "FALSE", "TRUE"),
  c( "noAllFr", "yesAllFr"))

nbcdCountyGapHa <-
  data.table(
    cast(
      data= statsNbcdCounty[, list( ha=sum( ha)),
        keyby= "state,county,gap,nbcd"],
      formula= state + county ~ gap + nbcd,
      value= "ha"),
    key= "state,county")

setnames(
  nbcdCountyGapHa,
  c( "FALSE_FALSE", "FALSE_TRUE", "TRUE_FALSE", "TRUE_TRUE"),
  c( "noAllHaN", "noAllHa", "yesAllHaN", "yesAllHa"))

nbcdCountyFrac <- 
  data.table(
    cast(
      data= statsNbcdCounty,
      subset= nbcd,
      formula= state + county ~ gap + nlcd,
      value= "frac"),
    key= "state,county")

setnames(
  nbcdCountyFrac,
  colnames( nbcdCountyFrac)[ -(1:2)],
  paste(
    str_replace(
      str_replace(
        colnames( nbcdCountyFrac)[ -(1:2)],
        "TRUE_", "yes"),
      "FALSE_", "no"),
    "Fr", sep= ""))

nbcdNullCountyFrac <- 
  data.table(
    cast(
      data= statsNbcdCounty,
      subset= !nbcd,
      formula= state + county ~ gap + nlcd,
      value= "frac"),
    key= "state,county")

setnames(
  nbcdNullCountyFrac,
  colnames( nbcdNullCountyFrac)[ -(1:2)],
  paste(
    str_replace(
      str_replace(
        colnames( nbcdNullCountyFrac)[ -(1:2)],
        "TRUE_", "yes"),
      "FALSE_", "no"),
    "FrN", sep= ""))

nbcdCountyHa <- 
  data.table(
    cast(
      data= statsNbcdCounty,
      subset= nbcd,
      formula= state + county ~ gap + nlcd,
      value= "ha",),
    key= "state,county")

setnames(
  nbcdCountyHa,
  colnames( nbcdCountyHa)[ -(1:2)],
  paste(
    str_replace(
      str_replace(
        colnames( nbcdCountyHa)[ -(1:2)],
        "TRUE_", "yes"),
      "FALSE_", "no"),
    "Ha", sep= ""))

nbcdNullCountyHa <- 
  data.table(
    cast(
      data= statsNbcdCounty,
      subset= !nbcd,
      formula= state + county ~ gap + nlcd,
      value= "ha",),
    key= "state,county")

setnames(
  nbcdNullCountyHa,
  colnames( nbcdNullCountyHa)[ -(1:2)],
  paste(
    str_replace(
      str_replace(
        colnames( nbcdNullCountyHa)[ -(1:2)],
        "TRUE_", "yes"),
      "FALSE_", "no"),
    "HaN", sep= ""))


nbcdCounty <-
  nbcdCountyAldb[ nbcdCountyAldbMeans]
nbcdCounty <-
  nbcdCounty[ nbcdCountyGapFrac][ nbcdCountyGapHa]
nbcdCounty <-
  nbcdCounty[ nbcdCountyFrac][ nbcdCountyHa]
nbcdCounty <-
  nbcdCounty[ nbcdNullCountyFrac][ nbcdNullCountyHa]

nbcdCounty <-
  nbcdCounty[, fips := paste( state, county, sep= "")]
nbcdCounty <-
  nbcdCounty[, state := NULL][, county := NULL]
setkey( nbcdCounty, fips)
setcolorder( nbcdCounty, order( colnames( nbcdCounty)))

nbcdCountyChar <-
  str_trim(
    format.df(
      nbcdCounty,
      cdec= sapply(
        colnames( nbcdCounty),
        function( x)
        ifelse(
          x == "fips", 0,
          ifelse(
            str_detect( x, "HaN?$"), 1,
            ifelse(
              str_detect( x, "FrN?$"), 3,
              1)))),
      numeric.dollar= FALSE,
      na.blank= TRUE))

write.csv(
  nbcdCountyChar,
  row.names= FALSE,
  file= "pad-us_nlcd/nbcdCounty.csv",
  quote= FALSE)

zip( "pad-us_nlcd.zip", "pad-us_nlcd/nbcdCounty.csv")

options(useFancyQuotes = FALSE)
cat(
  sapply(
    colnames( nbcdCounty),
    function( x) {
      dQuote(
        ifelse(
          x == "fips", "String(5)",
          ifelse(
            str_detect( x, "HaN?$"),
            "Real(10.1)",
            ifelse(
              str_detect( x, "FrN?$"),
              "Real(5.3)",
              "Real(5.1)"))))
    }),
  sep= ",",
  file= "pad-us_nlcd/nbcdCounty.csvt")
options(useFancyQuotes = TRUE)

ogr2ogr <-
  paste(
    "ogr2ogr -overwrite -progress -sql",
    sprintf(
      "\"select %s from cusaCountiesAea a",
      paste( colnames( nbcdCounty), collapse= ",")),
    "left join 'pad-us_nlcd/nbcdCounty.csv'.nbcdCounty b",
    "on a.GEOID10 = b.fips\"",
    "pad-us_nlcd/nbcdCounty.shp shp/cusaCountiesAea.shp")

system( ogr2ogr)

zip(
  "pad-us_nlcd.zip",
  list.files(
    path= "pad-us_nlcd",
    pattern= "^nbcdCounty\\.[a-z]{3}$",
    full.names= TRUE))
\end{verbatim}



\item \textbf{TODO} Work out a way to update changed files in the output .zip archive
\label{sec-4-1}%



\item \textbf{TODO} include version and environment info from the various tools
\label{sec-5-1-1}%


\item \textbf{TODO} copy software BibTeX entries to database
\label{sec-5-1-2}%

\end{itemize} % ends low level
\section{Plots}
\label{sec-4}




\begin{verbatim}

library( ggplot2)
library( scales)

## totalTonnes <-
##   statsNbcdCounty[, list( aldb= sum( aldb * ha, na.rm= TRUE)),
##                   keyby= "gap,nlcd"]

## totalTonnes[, labelY := aldb/2 + c(0, cumsum( aldb)[-length( aldb)])]

## totalTonnes[, pct := round( aldb / sum( aldb) *100, 1)]
## totalTonnes[, label := ifelse( pct >= 0.5,
##                 sprintf( "%s, %3.1f%%", nlcd, pct), "")]

## totalTonnes[, label := sprintf( "%s, %d%%", nlcd, pct)]



totalTonnes <-
  statsNbcdCounty[, list( aldb= sum( aldb * ha, na.rm= TRUE),
                         ha= sum( ha, na.rm= TRUE)),
                  keyby= "nlcd,gap"]
totalTonnes <-
  totalTonnes[, aldbFrac := aldb / sum( aldb)]
totalTonnes <-
  totalTonnes[, haFrac := ha / sum( ha)]

nlcdColors <-
  c(
    "11" = "#5475A8",
    "12" = "#FFFFFF",
    "21" = "#E8D1D1",
    "22" = "#E29E8C",
    "23" = "#FF0000",
    "24" = "#B50000",
    "31" = "#D2CDC0",
    "41" = "#85C77E",
    "42" = "#38814E",
    "43" = "#D4E7B0",
    "52" = "#DCCA8F",
    "71" = "#FDE9AA",
    "81" = "#FBF65D",
    "82" = "#CA9146",
    "90" = "#C8E6F8",
    "95" = "#64B3D5")

nlcdCovers <-
  c(
    "11" = "water",
    "12" = "ice",
    "21" = "dev open",
    "22" = "dev low",
    "23" = "dev med",
    "24" = "dev high",
    "31" = "barren",
    "41" = "deciduous",
    "42" = "evergreen",
    "43" = "mixed",
    "52" = "shrub",
    "71" = "grass",
    "81" = "pasture",
    "82" = "crop",
    "90" = "woody wet",
    "95" = "wetland")

nlcdMeta <-
  data.table(
    nlcd= factor( names( nlcdColors)),
    color= nlcdColors,
    cover= nlcdCovers,
    key= "nlcd")

totalTonnes[, list( gap, nlcd,
                   frac = sprintf( "%5.4f", frac)),
            key= "nlcd"][ nlcdMeta]

totalTonnes <-
  totalTonnes[, nlcd := reorder( factor(nlcd), aldbFrac, max)]
setkey( totalTonnes, nlcd)

## with( totalTonnes, reorder( factor(nlcd), frac, max))


( massFracPlot <-
 ggplot(
   totalTonnes,
   aes(
     x= nlcd,
     y= aldbFrac,
     color= gap )) +
 geom_point(
   size= 4) +
 scale_x_discrete(
   name= "NLCD 2001 v1", ## ) +
   labels= nlcdMeta[ J( levels( totalTonnes$nlcd))][, cover]) +
 ylab( "Total mass fraction") +
 scale_color_manual(
   values= c( ## "#8C510A",
     "#D8B365", ## 0xF6E8C3; 0xC7EAE5;
     "#5AB4AC" ## "#01665E"
     )) + 
 coord_flip() +
 labs( colour= "Protected") +
 theme_bw())


## massFracPlot +
##   scale_y_log10(
##     limits= c(0.003, 0.35),
##     breaks= c( 0.01, 0.02, 0.03, 0.1, 0.2, 0.3), 
##     labels= percent)

## last_plot() +
##   scale_y_log10(
##     limits= c(0.003, 0.35),
##     breaks= c( 0.01, 0.02, 0.03, 0.1, 0.2, 0.3), 
##     labels= percent)
\end{verbatim}
   


\begin{verbatim}

suppressWarnings(
  massFracPlot %+%
  as.data.frame( totalTonnes[ aldbFrac > 0.003]) +
  scale_x_discrete(
    ## breaks= nlcdCovers[ as.character( totalTonnes[ frac > 0.003]$nlcd)],
    labels= nlcdCovers[ as.character( totalTonnes[ aldbFrac > 0.003]$nlcd),
      drop= TRUE]))
\end{verbatim}

\includegraphics[width=.9\linewidth]{pad-us_nlcd/images/massFracPlot.png}


\begin{verbatim}

totalTonnes <-
  totalTonnes[, nlcd := reorder( factor(nlcd), haFrac, max)]
setkey( totalTonnes, nlcd)

( areaFracPlot <-
 ggplot(
   totalTonnes,
   aes(
     x= nlcd,
     y= haFrac,
     color= gap )) +
 geom_point(
   size= 4) +
 scale_x_discrete(
   name= "NLCD 2001 v1", ## ) +
   labels= nlcdMeta[ J( levels( totalTonnes$nlcd))][, cover]) +
 ylab( "Total area fraction") +
 scale_color_manual(
   values= c( ## "#8C510A",
     "#D8B365", ## 0xF6E8C3; 0xC7EAE5;
     "#5AB4AC" ## "#01665E"
     )) + 
 coord_flip() +
 labs( colour= "Protected") +
 theme_bw())

## suppressWarnings(
##   massFracPlot %+%
##   as.data.frame( totalTonnes) + ##[ haFrac > 0.003]) +
##  ylab( "Total area fraction") +
##   scale_x_discrete(
##     labels= nlcdCovers[ as.character( totalTonnes$nlcd), ##[ haFrac > 0.003]$nlcd),
##       drop= TRUE]))
\end{verbatim}

\includegraphics[width=.9\linewidth]{pad-us_nlcd/images/areaFracPlot.png}


\begin{verbatim}

zip(
  "pad-us_nlcd.zip",
  list.files( "pad-us_nlcd/images", "png$", full.names=TRUE))
file.copy(
  from= list.files( "pad-us_nlcd/images", full.names= TRUE), 
  to=   "~/Dropbox/Carbon Inventory/images",
  overwrite= TRUE)

file.copy(
  from= "pad-us_nlcd.zip",
  to=   "~/Dropbox/Carbon Inventory",
  overwrite= TRUE)
\end{verbatim}
\section{Appendix}
\label{sec-5}
\subsection{Software Environment}
\label{sec-5-1}




\begin{verbatim}
 GNU Emacs 24.3.50.1
 Copyright (C) 2013 Free Software Foundation, Inc.
\end{verbatim}



\begin{verbatim}
 Org-mode version 7.9.4 (7.9.4-elpa @ /home/nbest/.emacs.d/elpa/org-20130318/)
\end{verbatim}




\begin{verbatim}
R version 2.15.3 (2013-03-01)
Platform: x86_64-pc-linux-gnu (64-bit)

locale:
 [1] LC_CTYPE=en_US.UTF-8       LC_NUMERIC=C              
 [3] LC_TIME=en_US.UTF-8        LC_COLLATE=en_US.UTF-8    
 [5] LC_MONETARY=en_US.UTF-8    LC_MESSAGES=en_US.UTF-8   
 [7] LC_PAPER=C                 LC_NAME=C                 
 [9] LC_ADDRESS=C               LC_TELEPHONE=C            
[11] LC_MEASUREMENT=en_US.UTF-8 LC_IDENTIFICATION=C       

attached base packages:
[1] stats     graphics  grDevices utils     datasets  methods   base     

other attached packages:
 [1] rgdal_0.7-18     raster_1.9-70    sp_0.9-99        foreign_0.8-53  
 [5] ascii_2.1        doMC_1.2.5       multicore_0.1-7  iterators_1.0.6 
 [9] foreach_1.4.0    plyr_1.7.1       stringr_0.6.1    data.table_1.8.0

loaded via a namespace (and not attached):
[1] codetools_0.2-8 compiler_2.15.3 grid_2.15.3     lattice_0.20-13
[5] parallel_2.15.3 tools_2.15.3
\end{verbatim}




\begin{verbatim}

To cite package ‘raster’ in publications use:

  Robert J. Hijmans & Jacob van Etten (2012). raster: Geographic
  analysis and modeling with raster data. R package version 1.9-70.
  http://CRAN.R-project.org/package=raster

A BibTeX entry for LaTeX users is

  @Manual{,
    title = {raster: Geographic analysis and modeling with raster data},
    author = {Robert J. Hijmans & Jacob van Etten},
    year = {2012},
    note = {R package version 1.9-70},
    url = {http://CRAN.R-project.org/package=raster},
  }

ATTENTION: This citation information has been auto-generated from the
package DESCRIPTION file and may need manual editing, see
‘help("citation")’ .

To cite package ‘data.table’ in publications use:

  M Dowle, T Short and S Lianoglou (2012). data.table: Extension of
  data.frame for fast indexing, fast ordered joins, fast assignment,
  fast grouping and list columns.. R package version 1.8.0.
  http://CRAN.R-project.org/package=data.table

A BibTeX entry for LaTeX users is

  @Manual{,
    title = {data.table: Extension of data.frame for fast indexing, fast ordered joins,
fast assignment, fast grouping and list columns.},
    author = {M Dowle and T Short and S Lianoglou},
    year = {2012},
    note = {R package version 1.8.0},
    url = {http://CRAN.R-project.org/package=data.table},
  }

ATTENTION: This citation information has been auto-generated from the
package DESCRIPTION file and may need manual editing, see
‘help("citation")’ .
\end{verbatim}


\begin{verbatim}
gdalinfo --version
\end{verbatim}

\begin{verbatim}
 GDAL 1.9.2, released 2012/10/08
\end{verbatim}



\begin{verbatim}
grass64 --version
\end{verbatim}


\begin{verbatim}
GRASS GIS 6.4.2

Geographic Resources Analysis Support System (GRASS) is Copyright,
1999-2012 by the GRASS Development Team, and licensed under terms of the
GNU General Public License (GPL) version >=2.
 
This GRASS 6.4.2 release is coordinated and produced by the
GRASS Development Team with contributions from all over the world.

This program is distributed in the hope that it will be useful, but
WITHOUT ANY WARRANTY; without even the implied warranty of
MERCHANTABILITY or FITNESS FOR A PARTICULAR PURPOSE.  See the GNU
General Public License for more details.
\end{verbatim}

\end{document}
